\section{Introduction}

Quantum computing holds the potential to efficiently solve important problems that are too complex to tackle on classical computers.  For certain classes of problems, quantum algorithms offer an exponential speedup over their best-known classical counterparts.  Famous examples that illustrate the theoretical advantages of quantum computing include Grover's algorithm for searching through large unstructured databases \cite{grover1996fast} and Shor's algorithm for factoring large semi-prime numbers \cite{Shor_1997}.

However, today's quantum computing hardware is said to exist in the Noisy Intermediate Scale Quantum (NISQ) era, and it is still somewhat of an open question if any advantage over classical computing can be demonstrated on NISQ computers.  This is because current quantum computers are extremely sensitive to environmental perturbations, and most quantum algorithms have much higher resource requirements in terms of qubit count and fidelity than what is available in the near term.  Such perturbations are referred to as noise, and they cause decoherence of the fragile state of the quantum computer.  The NISQ era represents a crucial phase in quantum computing, where although quantum processors are still susceptible to noise and decoherence, they are also too large to be simulated classically. In this context, quantum algorithms tailored for NISQ devices have garnered significant attention. These algorithms aim to strike a delicate balance between harnessing quantum advantages and mitigating the impact of noise, making them practical for today's quantum hardware.  There is no guarantee that quantum computers will ever be stable enough to run algorithms such as those invented by Grover or Shor, so the development of powerful NISQ algorithms is crucial for demonstrating practical utility with quantum computers.

Among the diverse array of NISQ era algorithms, variational quantum algorithms have emerged as promising candidates. These algorithms combine the strengths of both quantum and classical computing, creating a hybrid approach that leverages quantum resources for specific tasks while employing classical methods for everything else.  By restricting usage of quantum processing to the classical computational bottlenecks within a hybrid algorithm, variational quantum algorithms can navigate the challenges posed by noise and decoherence, offering a path towards the realization of near-term practical quantum computations.  

This paper introduces one such variational quantum algorithm called the Variational Quantum Eigensolver (VQE).  VQE is particularly well-suited for solving eigenvalue problems, making it a versatile tool for applications in quantum chemistry, materials science, and optimization.  We discuss the original proposal of the VQE algorithm by Peruzzo et al. in 2014, as well as two recent VQE advancements that extend the applicability of the VQE framework.
\subsection{Mathematical Background}

\subsubsection{Quantum Bits}
In classical computing, information is processed using bits, which can exist in one of two states: 0 or 1. Quantum computing, on the other hand, processes quantum bits (called qubits) that exist in a superposition of states. This means that qubits exist in a linear combination of $\ket{0}$ and $\ket{1}$ and are represented by a quantum state $\ket{\psi} = \alpha\ket{0} + \beta\ket{1}$, where $\alpha$ and $\beta$ are complex numbers, and $\mid\alpha\mid^2 + \mid\beta\mid^2 = 1$.  The probability of measuring the state $\ket{\psi}$ to be $\ket{0}$ is $\mid\alpha\mid^2$, and the probability of measuring the state $\ket{\psi}$ to be $\ket{1}$ is $\mid\beta\mid^2$.  While $\alpha$ and $\beta$ may look like probabilities, their ability to take on negative and complex values leads to interference patterns in the amplitudes of multi-qubit systems that could not exist with strictly positive and real probabilities.

The power of qubits lies in their ability to be entangled, a phenomenon where the state of one qubit becomes dependent on the state of another, regardless of the physical distance between them. Entanglement allows quantum computers to process information in a massively parallel way, offering the potential for exponential speedup in solving certain problems.  Whereas a general two qubit system would take the form $\ket{\psi} = \alpha\ket{00} + \beta\ket{01} + \gamma\ket{10} + \delta\ket{11}$, an example of an entangled two qubit system could look like $\ket{\psi} = \alpha\ket{00} + \beta\ket{11}$.  In this case, although each qubit is individually in a superposition of 0 and 1, knowledge of one qubit's state immediately gives knowledge of the other qubit's state because they are always the in the same state: both 0 or both 1.  Another example of an entangled two-qubit state is $\ket{\psi} = \alpha\ket{01} + \beta\ket{10}$.  Here, we know that the two qubits are always in opposite states.

\subsubsection{Dirac Notation}
Quantum states are represented using Dirac notation, a mathematical framework developed by physicist Paul Dirac. In this notation, a quantum state $\ket{\psi}$ is represented as a column vector called a ket vector. For example, the quantum state of a qubit in superposition can be expressed as $\ket{\psi} = \alpha\ket{0} + \beta\ket{1} = \begin{bmatrix} \alpha & \beta \end{bmatrix}^T$.

The corresponding bra vector, denoted as $\bra{\psi}$, is the conjugate transpose of the ket vector. If $\ket{\psi} = \begin{bmatrix} \alpha & \beta \end{bmatrix}^T$, then $\bra{\psi} = \alpha^*\bra{0} + \beta^*\bra{1} = \begin{bmatrix} \alpha^* & \beta^* \end{bmatrix}$, where $^*$ denotes the complex conjugate.

Extending to higher dimensions, the quantum state of an n-qubit quantum computer is a ket vector of $2^n$ complex-valued coefficients.


\subsubsection{The Hamiltonian Operator}
The Hamiltonian operator $H$ plays a central role in quantum mechanics. It represents the total energy of a quantum system and is crucial for describing the evolution of the system over time. For a quantum system with a Hamiltonian operator $H$, the time-dependent Schrödinger equation is given by

\begin{equation*}
-\frac{\hbar^2}{2m} \nabla^2 \psi(\vec{r}, t) + V(\vec{r}, t) \Psi(\vec{r}, t) = i\hbar \frac{\partial \Psi(\vec{r}, t)}{\partial t}
\end{equation*}
which reduces to $H\ket{\psi} = E\psi$
where $\hbar$ is the reduced Planck constant and $E$ is the energy of the state.  It is clear in this form that the Schrödinger equation is an eigenvalue equation relating an eigenvector of a Hamiltonian to its energy.  

The eigenvectors $\ket{\psi}$ are called eigenstates, the lowest energy eigenstate of a Hamiltonian is called the ground state, and all other eigenstates are excited states.

\subsubsection{Hermitian Operators and Expectation Values}
In quantum mechanics, an observable is a property of a quantum state whose value can be determined by performing a certain physical measurement on the state.  Common examples include position, momentum, and energy measurements.  Because the measurement outcome must always yield a real value (a complex valued position measurement, for example, would be difficult to interpret physically), observables correspond to Hermitian matrices which have the property that they only admit real eigenvalues.

Pauli matrices are a set of 2-by-2 matrices that are both unitary and Hermitian.  Importantly for the Variational Quantum Eigensolver, tensor products of Pauli matrices form an orthonormal basis for the space of all Hermitian matrices.  In other words, any Hermitian matrix can be expressed as a real linear combination of tensor products of Pauli matrices.

The expectation value of an observable represents the average value that one would expect to see over repeated measurements of that observable on identically prepared systems. For an observable with associated matrix operator $A$, the expectation value with respect to state $\ket{\psi}$ is $\expval{A} = \bra{\psi}A\ket{\psi}$.  The calculation is performed by multiplying the matrix $A$ with the ket vector $\ket{\psi}$, then taking the inner product with the bra vector $\bra{\psi}$.  The expectation value of the Hamiltonian operator is particularly important, as it gives the average energy of the system with respect to a given state $\ket{\psi}$.


\subsection{Literature Review}

When it comes to solving large-scale eigenvalue problems and optimization problems, one particularly promising candidate for NISQ era quantum advantage is the Variational Quantum Eigensolver (VQE) proposed in \cite{peruzzo_variational_2014}.  VQE belongs to a class of hybrid algorithms that combine quantum and classical resources to solve complex problems.  By incorporating a classical optimization subroutine, VQE efficiently calculates the minimum eigenvalue of a given Hamiltonian operator as well as its corresponding eigenvector.  In modeling physical systems, this corresponds to solving for the ground state energy of the system, and in optimization problems this can map to solving for the optimal solution.
    
With VQE, we are interested in finding the eigenstates $\ket{\psi_i}$ and corresponding eigenvalues $\lambda_i$ of a Hamiltonian $\textit{H}$.  In particular, \cite{peruzzo_variational_2014} describes how to find the minimum eigenvalue and corresponding eigenvector in two steps.  First, we use the fact that tensor products of Pauli matrices form a basis for the space of Hermitian operators to implement Hamiltonian Averaging with an $n$ qubit quantum processor.  In this step, VQE efficiently decomposes a $2^n$ by $2^n$ Hamiltonian into a real linear combination of tensor products of Pauli operators.  The expectation value of each of these tensor products is then computed with respect to a parameterized input state $\ket{\psi_{\theta}}$.  The state $\ket{\psi_{\theta}}$ is prepared by applying a sequence of parameterized unitary matrices to a given reference state that is usually the uniform superposition state over all $2^n$ computational basis states.  Second, a classical processor is used to sum these weighted Pauli expectation values, returning the expectation value $\expval{H_\theta}  \equiv \bra{\psi_\theta}\textit{H}\ket{\psi_\theta}$.  Still using a classical processor, a nonlinear optimizer such as the Nelder-Mead simplex method \cite{Nelder_Mead_1965} is used to determine new values of $\theta$ that minimize $\expval{H_\theta}$.  A new input state is then prepared on the quantum processor, and the process repeats until $\expval{H_\theta}$ converges to its lowest eigenvalue.  The corresponding eigenstate can then easily be reconstructed using the values of $\theta$.

In the time since VQE was originally introduced, several extensions have been proposed to systematically find excited states (the other eigenvalues) of a Hamiltonian with minimal extra cost.  One such extension is Variational Quantum Deflation (VQD) \cite{higgott_variational_2019}.  VQD exploits the fact that Hermitian matrices admit a complete set of orthogonal eigenvectors to iteratively calculate eigenvalues in order from least to greatest.  In essence, VQE is first used to calculate the ground state.  Then, for each subsequent eigenvalue calculation, VQD is used to iteratively minimize the overlap between the parameterized state and the span of previously calculated eigenstates.  While this method has shown promise for finding lower eigenstates, errors tend to build up in calculations for higher eigenstates due to inexact approximations of eigenstates in previous iterations.

Another approach to extending VQE beyond ground state calculations is the Subspace-Search Variational Quantum Eigensolver (SSVQE) \cite{nakanishi_subspace-search_2019}.  SSVQE searches a low energy subspace by supplying orthogonal input vectors $\ket{\Psi_i}$ and utilizes the preservation of orthogonality under unitary transformations to ensure the orthogonality of output states from the state preparation step of VQE.  A minimization routine is run to find the orthogonal transformation that maps the span of the k input vectors to the eigenspace of the first k eigenvectors.  Then, the k\textsuperscript{th} eigenstate is calculated by finding the highest energy state in the low energy subspace spanned by the transformed input vectors.