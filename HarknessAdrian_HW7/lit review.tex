\subsection{Literature Review}

When it comes to solving large-scale eigenvalue problems and optimization problems, one particularly promising candidate for NISQ era quantum advantage is the Variational Quantum Eigensolver (VQE) proposed in \cite{peruzzo_variational_2014}.  VQE belongs to a class of hybrid algorithms that combine quantum and classical resources to solve complex problems.  By incorporating a classical optimization subroutine, VQE efficiently calculates the minimum eigenvalue of a given Hamiltonian operator as well as its corresponding eigenvector.  In modeling physical systems, this corresponds to solving for the ground state energy of the system, and in optimization problems this can map to solving for the optimal solution.
    
With VQE, we are interested in finding the eigenstates $\ket{\psi_i}$ and corresponding eigenvalues $\lambda_i$ of a Hamiltonian $\textit{H}$.  In particular, \cite{peruzzo_variational_2014} describes how to find the minimum eigenvalue and corresponding eigenvector in two steps.  First, we use the fact that tensor products of Pauli matrices form a basis for the space of Hermitian operators to implement Hamiltonian Averaging with an $n$ qubit quantum processor.  In this step, VQE efficiently decomposes a $2^n$ by $2^n$ Hamiltonian into a real linear combination of tensor products of Pauli operators.  The expectation value of each of these tensor products is then computed with respect to a parameterized input state $\ket{\psi_{\theta}}$.  The state $\ket{\psi_{\theta}}$ is prepared by applying a sequence of parameterized unitary matrices to a given reference state that is usually the uniform superposition state over all $2^n$ computational basis states.  Second, a classical processor is used to sum these weighted Pauli expectation values, returning the expectation value $\expval{H_\theta}  \equiv \bra{\psi_\theta}\textit{H}\ket{\psi_\theta}$.  Still using a classical processor, a nonlinear optimizer such as the Nelder-Mead simplex method \cite{Nelder_Mead_1965} is used to determine new values of $\theta$ that minimize $\expval{H_\theta}$.  A new input state is then prepared on the quantum processor, and the process repeats until $\expval{H_\theta}$ converges to its lowest eigenvalue.  The corresponding eigenstate can then easily be reconstructed using the values of $\theta$.

In the time since VQE was originally introduced, several extensions have been proposed to systematically find excited states (the other eigenvalues) of a Hamiltonian with minimal extra cost.  One such extension is Variational Quantum Deflation (VQD) \cite{higgott_variational_2019}.  VQD exploits the fact that Hermitian matrices admit a complete set of orthogonal eigenvectors to iteratively calculate eigenvalues in order from least to greatest.  In essence, VQE is first used to calculate the ground state.  Then, for each subsequent eigenvalue calculation, VQD is used to iteratively minimize the overlap between the parameterized state and the span of previously calculated eigenstates.  While this method has shown promise for finding lower eigenstates, errors tend to build up in calculations for higher eigenstates due to inexact approximations of eigenstates in previous iterations.

Another approach to extending VQE beyond ground state calculations is the Subspace-Search Variational Quantum Eigensolver (SSVQE) \cite{nakanishi_subspace-search_2019}.  SSVQE searches a low energy subspace by supplying orthogonal input vectors $\ket{\Psi_i}$ and utilizes the preservation of orthogonality under unitary transformations to ensure the orthogonality of output states from the state preparation step of VQE.  A minimization routine is run to find the orthogonal transformation that maps the span of the k input vectors to the eigenspace of the first k eigenvectors.  Then, the k\textsuperscript{th} eigenstate is calculated by finding the highest energy state in the low energy subspace spanned by the transformed input vectors.